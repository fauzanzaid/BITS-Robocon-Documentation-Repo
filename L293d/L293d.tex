\documentclass{article}

\usepackage{color}
\usepackage{xcolor}
\usepackage{graphicx}
\graphicspath{{/home/fauzan/Documents/BITS/Robocon/Documentation/Binaries/L293d/Resources/Images/}}
\usepackage{epstopdf}
\usepackage{listings}
\usepackage[a4paper]{geometry}
\usepackage{caption}
\usepackage{subcaption}
% \usepackage{menukeys}

% For tilde
\usepackage{textcomp}

\usepackage[driverfallback=dvipdfm, pdfborder={1 1 0.75}]{hyperref}

\DeclareGraphicsExtensions{.pdf,.png,.jpg}


\definecolor{lgray}{rgb}{0.8,0.8,0.8}
\definecolor{dgrin}{rgb}{0,0.6,0}
\definecolor{ngray}{rgb}{0.5,0.5,0.5}
\definecolor{nmauv}{rgb}{0.58,0,0.82}


\lstset{frame=tb,
	aboveskip=3mm,
	language=C++,
	belowskip=3mm,
	showstringspaces=false,
	columns=flexible,
	basicstyle={\small\ttfamily},
	numbers=left,
	numberstyle={\small\ttfamily\color{ngray}},
	keywordstyle=\color{blue},
	commentstyle=\color{dgrin},
	stringstyle=\color{nmauv},
	backgroundcolor=\color{lgray},
	breaklines=true,
	frame=single,
	breakatwhitespace=true,
	tabsize=4
}
\renewcommand\lstlistingname{Snippet}
\renewcommand\lstlistlistingname{Snippet}
\def\lstlistingautorefname{Snp.}

\title{L293D Dual H bridge motor driver}
\author{Fauzan}

\setlength{\fboxsep}{2pt}

\begin{document}

\newcommand{\inlncd}[1]{\colorbox{lgray}{\texttt{#1}}}
\newcommand{\icpin}[2]{\setlength{\fboxsep}{0pt}\fbox{\texttt{\colorbox{lgray}{\strut{}\,PIN~#1\,}\,#2\,}}\setlength{\fboxsep}{2pt}}


\maketitle

% \tableofcontents

% \newpage

\section{Introduction}

	\begin{figure}[h]
		\centering
		\includegraphics[width=0.3\textwidth]{L293d.jpeg}
	\end{figure}

	The L293D is a dual H bridge motor drive. It is also described as a quadruple half H bridge motor driver. It is 16 pin IC. The D stands for for the inbuilt kick back diodes that can shield the IC from inductive kick backs.

	The IC can supply 600~mA per channel, and can manage 1200~mA for short intervals. It can drive motors with voltage from 4.5 to 36V.

	The IC can drive four resistive loads, two DC motors bidirectionally, or one stepper motor at a time.

\newpage

\section{Pins}

	\begin{figure}[h]
		\centering
		\includegraphics[width=0.8\textwidth]{L293d_pin}
		\caption{Pin layout of an L293D}
	\end{figure}

	All ICs have a notch on one side to identify the top. Pins are then numbered anti clockwise from the top left.

	\begin{table}[h]
	\large
	\caption{Pin description}
		\begin{center}
		\begin{tabular}{|c|c|c|l|}
			\hline
			Pin name & Pin Number & Type & Description \\ \hline \hline
			1,2EN & 1 & Input & Enable driver channels 1 and 2 (active high) \\ \hline
			\textless{}1:4\textgreater{}A & 2, 7, 10, 15 & Input & Driver inputs (Non inverting) \\ \hline
			\textless{}1:4\textgreater{}Y & 3, 6, 11, 14 & Output & Driver Outputs \\ \hline
			3,4EN & 9 & Input & Enable driver channels 1 and 2 (active high) \\ \hline
			Ground & 4, 5, 12, 13 & -- & Ground and heat sink \\ \hline
			V\textsubscript{CC1} & 16 & -- & 5V supply for internal logic \\ \hline
			V\textsubscript{CC2} & 8 & -- & Power supply for outputs \\ \hline
		\end{tabular}
		\end{center}
	\end{table}

\newpage

\section{Using the IC}

	Make the following power connections

	\begin{enumerate}
		\item Connect \icpin{16}{V\textsubscript{CC1}} to 5V.
		\item Connect \icpin{4/5/12/13}{GND} to ground.
		\item Connect \icpin{16}{V\textsubscript{CC2}} to rated power supply for driving the load.
	\end{enumerate}

	All four channels are currently disabled. When \icpin{1}{EN1,2} and/or \icpin{9}{EN3,4} are \texttt{LOW}, the corresponding drivers are off and respective outputs, \icpin{3/6}{1Y/2Y} and/or \icpin{11/14}{3Y/4Y} are off and in high impedance state.

	Before using any driver, the corresponding enable pin should  be input logic \texttt{HIGH}.

	Suppose, \icpin{1}{EN1.2} is input \texttt{HIGH}. Now channels 1 and 2 are enabled. If you now input \texttt{HIGH} to \icpin{2}{1A}, then \icpin{3}{1Y} will output \texttt{V\textsubscript{CC2}}. If you ground \icpin{2}{1A}, then \icpin{3}{1Y} will also be grounded. Same affect will be observed with the second channel.

	You can similarly enable the third and fourth channel by giving \texttt{HIGH} input to \icpin{9}{EN3,4}.\@

	Analog voltages can be simulated by giving PWM to the channel inputs.

\newpage
	
\section{Driving two motors}

	Let us now wire up two motors, four potentiometers, an L293D and an Arduino Uno. With the help of pots, we will generate a voltage between 0 and 5V. We can not directly apply this voltage at the driver inputs as the L293D is a digital IC. It can only take either \texttt{HIGH} or \texttt{LOW} as input at \icpin{1A/2A/3A/4A}{2/7/10/15}. We can use PWM here. The Arduino will read the analog voltages and give PWM signals as output. This PWM can now be fed to the L293D, and will give a PWM output.\@

	Carefully trace and understand all the connections from the IC in the below circuit. Make sure you understand why the two ground rails have been shorted.


	\begin{figure}[!h]
		\centering
		\includegraphics[width=0.6\textwidth]{circuits_io_2467280.png}

		Simulate the circuit -- \url{https://circuits.io/circuits/2467280}
		\caption{Circuit to drive two DC motors}
	\end{figure}

	Here's the code:
	
	\lstinputlisting{./Resources/Codes/L293d_pot/L293d_pot.ino}

	Try to write code that uses just one pot per motor.

\section{Further Reading}

	Data sheet \url{http://www.ti.com/lit/ds/slrs008d/slrs008d.pdf}


\end{document}